\section{Introduction}

In this chapter, the general formulation used to setup mechanical systems so as to make them amenable to analysis with stochastic averaging is given. The theory of non-standard stochastic averaging is then presented.

The starting point is a general form for the equations of the dynamical systems that shall be averaged. The results of stochastic averaging based on the martingale problem are then given. The chapter concludes by giving a precise characterization of the Markov process that describes the evolution of the reduced stochastic process. This characterization also provides a precise definition of the features of the domain of the reduced Markov process.

% Hamiltonian equations of motion

% Conserved quantities (resonance)

\section{Theoretical Results of Stochastic Averaging}

The results given in this section provide the key formulas that enable the application of stochastic averaging theory to mechanical systems.

Note that rigorous proofs for the results given in this section have not been published. As such, the formulas given in this section should be taken as conjectures. These conjectures are based on two sources: (i) \citet{namachchivaya01:_unified_approac_noisy_nonlin_mathieu_type_system}, where systems having the same form as those in this thesis are analyzed, save for the fact that the problems in this thesis are reduced from four dimensions to two whereas in the reference the reduction is from two dimensions to one and (ii) \citet{freidlin04:_diffus}. In the second reference, the dimensions match those of this thesis, however averaging is performed for systems with two timescales, not three as is the case with the mechanical systems in this thesis.

The general form of the un-averaged systems in this thesis is:
\[
dZ_t^\epsilon = \frac{1}{\epsilon^2} b^0(Z_t^\epsilon) dt + \frac{1}{\epsilon} b^1(Z_t^\epsilon) dt + b^2(Z_t^\epsilon) dt + \sigma(Z_t^\epsilon) dW_t
\]
The generator of this process is
\[
\gen
\]
%%% Local Variables: 
%%% mode: latex
%%% TeX-master: "main"
%%% End: 
